%include common packages and settings
\usepackage{etex} %эта магическая херь избавляет от переполнения регистров TeX а!!!

\mode<article>{\usepackage{fullpage}}
\mode<presentation>{
    \usetheme{Madrid} %%Boadilla,Madrid,AnnArbor,CambridgeUS,Malmoe,Singapore,Berlin
    \useoutertheme{shadow}
} 

\usepackage[utf8]{inputenc}
\usepackage[russian]{babel}
\usepackage{indentfirst}
\usepackage{graphicx}

\usepackage{amsmath}
\usepackage{amsfonts}
\usepackage{amsthm}

%\date{Место презентации \\(\today)}
\author[М.~М.~Шихов]{Михаил Шихов \\ \texttt{\underline{m.m.shihov@gmail.com}}}


\title[Вяжем полотно кастинговой сети <<клиньями>>]{Полотно кастинговой сети}

\begin{document}

\mode<article>{\maketitle\tableofcontents}
\frame<presentation>{\titlepage}
\begin{frame}<presentation>
    \frametitle{Содержание}
    \tableofcontents
\end{frame}

\begin{frame}
    \frametitle{Что делаем?}

	\begin{itemize}
		\item Разберемся в общих принципах вязания круглого сетевого полотна.
		\item Поймем через сколько обычных рядов нужно вязать ряд с добавками и сколько в этом ряду этих самых добавок должно быть. 
		\item Свяжем несколько маленьких полотен в качестве примера и посмотрим, что получается.
		\item Выведем основные формулы.
	\end{itemize}
\end{frame}


\section{Как вязать круглое сетевое полотно?}

\begin{frame}
    \frametitle{Ячейка сети --- штука подвижная}
    \framesubtitle{Вытягиваясь по ширине, она уменьшается по высоте (и наоборот)}

    \begin{center}
        \includegraphics[width=0.95\textwidth]{figs/cells-simple}
    \end{center}
\end{frame}


\subsection{Вяжем одним циллиндром без добавок}

\begin{frame}
    \frametitle{Если связать всё полотно одинм циллиндром}
    \framesubtitle{При этом верхнюю часть просто <<стянуть>> к горловине}

    \begin{center}
        \includegraphics[width=0.95\textwidth]{figs/by-one-cylinder}
    \end{center}
\end{frame}

\begin{frame}
    \frametitle{В <<стянутом>> циллиндре}
    \framesubtitle{Ячейки по краям полотна будут раскрыты, а у центра --- практически закрыты}

    \begin{center}
        \includegraphics[width=0.5\textwidth]{figs/cells-in-cylinder-mesh}
    \end{center}
\end{frame}

\begin{frame}
    \frametitle{Достоинства и недостатки}
	
	Достоинства:
    \begin{itemize}
        \item Не надо уметь вязать добавки.
    \end{itemize}
	
	Недостатки:
    \begin{itemize}
        \item Жуткая <<мотня>> из ячей в центре будет мешать забросу.
    \end{itemize}
	
	Надо подумать:
    \begin{itemize}
        \item Ячейки, <<закрытые>> в центре --- это достоинство или недостаток?
		\begin{itemize}
			\item Центр сети массивнее в полете.
			\item Центр сети оказывает большее сопротивление воде, когда сеть тонет (парашют лучше?).
		\end{itemize}		
    \end{itemize}
\end{frame}


\subsection{Вяжем <<Клиньями>>}

\begin{frame}
    \frametitle{Если взять циллиндр и вырезать лишнее}
    \framesubtitle{Получатся клинья, в которых ячейки раскрыты равномерно}

    \begin{center}
        \includegraphics[width=0.43\textwidth]{figs/wedges}
    \end{center}
\end{frame}

\begin{frame}
    \frametitle{Сложив <<клинья>> на плоскость,}
    \framesubtitle{получим полотно с равномерным раскрытием ячеек}

    \begin{center}
        \includegraphics[width=0.43\textwidth]{figs/wedges-mesh-cells}
    \end{center}
\end{frame}

\begin{frame}
    \frametitle{Выводы}

	Когда полотно вяжется <<клиньями>>:
    \begin{itemize}
        \item Работы в два раза меньше, чем связать полотно одним циллиндром.
        \item Ячейки в полете получаются равномерно раскрытыми.
		\item Клин легко формируется добавлеинем ячеи на ребре.
        \item Как будет видно, схема вязания клиньями по кругу очень простая и регулярная. Ошибиться тут сложно.
    \end{itemize}
\end{frame}


\subsection{Вяжем полотно из нескольких циллиндров}

\begin{frame}
    \frametitle{Вяжем несколькими циллиндрами}
    \framesubtitle{Чем меньше высота составляющих циллиндров, тем равномернее раскрытие ячей}

    \begin{center}
        \includegraphics[width=0.43\textwidth]{figs/by-many-cylinders}
    \end{center}
\end{frame}

\begin{frame}
    \frametitle{Полусфера несколькими циллиндрами}

    \begin{center}
        \includegraphics[width=0.35\textwidth]{figs/half-sphere}
    \end{center}
\end{frame}


\section{Как правильно делать добавки?}

\begin{frame}
    \frametitle{Как вязать добавку?}

%TODO    \begin{center}
%            \includegraphics[width=0.35\textwidth]{figs/addition-how-to}
%        \end{center}
\end{frame}


\subsection{Схемы вывязывания <<клина>>}

\begin{frame}
    \frametitle{Чтобы вязать <<клиньями>>}

	надо определить:
    \begin{center}
        \item через сколько обычных рядов (провязаных циллиндром) нужно вязать ряд с добавками?
		\item сколько в ряду c добавками этих самых добавок должно быть?
    \end{center}
\end{frame}

\begin{frame}
    \frametitle{Чтобы вязать <<клиньями>>}

	надо определить:
    \begin{center}
        \item через сколько обычных рядов (провязаных циллиндром) нужно вязать ряд с добавками?
		\item сколько в ряду c добавками этих самых добавок должно быть?
    \end{center}
\end{frame}


%\subsection{Сколько добавок делать на ряд?}
%\subsection{Что получается на практике?}
%
%\section{Сколько труда придется вложить?}


\end{document}